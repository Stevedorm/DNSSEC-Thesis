\chapter{Background}

\section{DNS Research}
Firstly, I think a bit of background about DNS helps to understand why it became waht it is today. DNS started 
as ARPANET, which mapped names to addresses using a host file that was distributed
to all entities whenever changes occurred. As it may seem, this system became rapidly unsustainable once there
were over 100 networked entities, which led to DNS today.\cite{bind-9}

In my research about DNS, I learned a lot about what happens behind the scenes with DNS Servers.
Cloudflare called it, "DNS is like the phone book of the internet." \cite{cloudflare-dns} All network systems operate with network addresses, such as IPv4 and IPv6. 
More or less, DNS translates domain names, like www.example.com, to IP addresses, like 127.0.0.1, so browsers can load internet resources.

DNS naming system is organized as a tree structure made up of multiple levels and naturally creates a distributed system\cref{fig:dns}.

\begin{figure}[htbp]
  \centering
  \includegraphics[width=0.8\textwidth]{figures/dns.png}
  \caption{Test}
  \label{fig:dns}
\end{figure}




\section{VM Research}

I have never dealt directly with Virtual Machines before, and building a network of them was something I had never really thought about. However, the thought excited me 
and it seemed like a great opportunity to learn a new skill.